\documentclass[a4paper,11pt]{article}
\usepackage[margin=2cm]{geometry}

\usepackage[titletoc,toc,title,page]{appendix}
\usepackage[nodayofweek]{datetime}
\usepackage{cite}
\usepackage{graphicx}
\longdate

\usepackage{titlesec}
\usepackage{hyperref}
\usepackage{fancyhdr}
\pagestyle{fancyplain}
\fancyhf{}
\lhead{\fancyplain{}{M.Sc.\ Group Project Report}}
\rhead{\fancyplain{}{\today}}
\cfoot{\fancyplain{}{\thepage}}

\begin{document}

\subsection{Pattern Recognition}

The purpose of the pattern recognition module is to discern recurring spike patterns within the output from the CSTMD1 module, with each pattern recognition neuron becoming selective to one pattern. In order to model these neurons, we initially replicated experiments conducted by Masquelier et al. \cite{stdp2} \cite{stdp1}. These experiments showed that spike Response Model (SRM) leaky integrate-and-fire neurons could successfully recognised input patterns based on sample input generated from a Poisson process. A single of these neurons is able to successfully recognise a recurring pattern within background noise and a network of them is able to do so for multiple patterns. This behaviour is achieved by modulating the weights of the pattern recognition neuron's synaptic connections to its afferents using spike timing dependent plasticity (STDP). STDP uses Long Term synaptic Potentiation (LTP) to reinforce connections with afferents that fired shortly before the postsynaptic neuron, and Long Term Depression (LTD) to weaken those with afferents that fired shortly after. Given that the input patterns occur within random noise, STDP will favour those afferents that participate in the pattern, as every time the pattern manifests itself, they will consistently fire in a given order. Within 15 seconds of simulation time, the pattern recognition neuron becomes selective to the pattern, and continuously reinforces the connections of the afferents that fired slightly before it discharged. Hence with every manifestation of the pattern, the neuron is more likely to fire earlier within it, effectively signalling its beginning.


In order to allow for different pattern recognition neurons to become selective to different patterns, we followed Masquelier et al. (2009), connecting a network of pattern recognition neurons to the sample inputs, introducing inhibitory connections amongst the post-synaptic neurons. This allows for a single postsynaptic neuron to become selective to one pattern and to inhibit other postsynaptic neurons from becoming selective to that pattern, thus allowing them to bind to other patterns in the input. 

We then extended the module so that the neurons can be easily adapted to recognise input with varying properties such as average firing rate, number of afferents, frequency of pattern appearance, amongst others. This implementation is able to recognise patterns output from our CSTMD1 neurons and measures the effectiveness of the pattern recognition neurons by tracking key information such as true-positive, false-positive and true-negative spike incidences.


\end{document}
