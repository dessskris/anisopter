\documentclass[a4paper,11pt]{article}
\usepackage[margin=2cm]{geometry}

\usepackage[titletoc,toc,title,page]{appendix}
\usepackage[nodayofweek]{datetime}
\usepackage{cite}
\usepackage{graphicx}
\longdate

\usepackage{titlesec}
\usepackage{hyperref}
\usepackage{fancyhdr}
\pagestyle{fancyplain}
\fancyhf{}
\lhead{\fancyplain{}{M.Sc.\ Group Project Report}}
\rhead{\fancyplain{}{\today}}
\cfoot{\fancyplain{}{\thepage}}

\begin{document}

\subsection{Pattern Recognition}

The purpose of the pattern recognition module is to discern recurring spike patterns within the output from the CSTMD1 module, with each pattern recognition neuron becoming selective to one pattern. Those patterns encode information about the velocity and direction of the target observed in the visual field of the dragonfly.\par

	In order to model these neurons, we initially replicated experiments conducted by Masquelier et al. \cite{stdp2} \cite{stdp1}. These experiments showed that spike Response Model (SRM) leaky integrate-and-fire neurons could successfully recognise input patterns based on sample input generated from a Poisson process. A single of these neurons is able to successfully recognise a recurring pattern within background noise and a network of them is able to do so for multiple patterns. This behaviour is achieved by modulating the weights of the pattern recognition neuron's synaptic connections to its afferents using spike timing dependent plasticity (STDP). STDP uses Long Term synaptic Potentiation (LTP) to reinforce connections with afferents that fired shortly before the postsynaptic neuron, and Long Term Depression (LTD) to weaken those with afferents that fired shortly after. Given that the input patterns occur within random noise, STDP will favour those afferents that participate in the pattern, as every time the pattern manifests itself, they will consistently fire in a given order. Within 15 seconds of simulation time, the pattern recognition neuron becomes selective to the pattern, and continuously reinforces the connections of the afferents that fired slightly before it discharged. Hence with every manifestation of the pattern, the neuron is more likely to fire earlier within it, effectively signalling its beginning.\par

	In order to allow for different pattern recognition neurons to become selective to different patterns, we followed Masquelier et al. (2009), connecting a network of pattern recognition neurons to the sample inputs, introducing inhibitory connections amongst the post-synaptic neurons. This allows for a single postsynaptic neuron to become selective to one pattern and to inhibit other postsynaptic neurons from becoming selective to that pattern, thus allowing them to bind to other patterns in the input.\par
	
	In a more extended model, post-synaptic neurons can become selective to part of a specific pattern. Masquelier et al. (2009) showed that given a 50ms spike pattern, up to 3 different neurons can fire within a single pattern thus identifying the beginning, middle and end of it. By adjusting the level of the mutual inhibition the number of the firing post-synaptic neurons can be increased or reduced depending the goal of the simulation. This extended model, although not used in our end product, it initiated long discussions about its potential in this project. During the time that the action selection module was not progressing as expected, we considered the alternative of using the aforementioned model to create a multiple level pattern recognition mechanism. The goal would be to investigate the extent of information regarding the velocity of the target (encoded in the visual input) that could be properly decoded by this multilevel pattern recognition mechanism. Despite the fact that we could have obtained results valuable to the research community, we were well aware that such a decision would result in a significant deviation from our original goals. Therefore, we kept our focus on the action selection module instead.\par

	Once the pattern recognition module was built, we extended it so that the neurons can be easily adapted to recognise input with varying properties such as average firing rate, number of afferents, frequency of pattern appearance, amongst others. This implementation is able to recognise patterns output from our CSTMD1 neurons and measures the effectiveness of the pattern recognition neurons by tracking key information such as true-positive, false-positive and true-negative spike incidences.


\end{document}
