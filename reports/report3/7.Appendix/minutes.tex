\documentclass{article}

\usepackage{paralist}
\usepackage[colorlinks=true]{hyperref}
\usepackage{circuitikz}
\usepackage{textcomp}
\usepackage{mathcomp}
\usepackage{fullpage}
\usepackage{bm}
\usepackage{amsmath,amssymb,amsthm,enumitem}

\title{Group Project Minutes}
\author{Logged by: Panagiotis Almpouras}

\begin{document}
\maketitle
\subsubsection*{This is a summary of the meeting minutes recorded throughout the project.}
\maketitle
\section*{16 January 2015}
\subsection*{Attendance}
\begin{compactenum}
\item Juan Carlos Farah (JCF)
\item Christos Kaplanis (CK)
\item Erik Grabljevec (EG)
\item Panagiotis Almpouras (PA)
\item Ioannis Kasidakis (IK)
\item Zafeirios Fountas (ZF)
\item Pedro Martínez Mediano(PMM)
\end{compactenum}

\subsection*{Summary}
This meeting was mainly focused on dscussing the way the group would tackle administrative tasks. The decision about the split of the groups was finalised and initial tasks (mainly background reading) was assigned to the two groups. The members were split in a group of three (JCF, EG, PA) that would focus on the pattern recognition and a group of two (CK, IK) that would focus on the action selection.

\maketitle
\section*{21 January 2015}
\subsection*{Attendance}
\begin{compactenum}
\item Juan Carlos Farah (JCF)
\item Christos Kaplanis (CK)
\item Erik Grabljevec (EG)
\item Panagiotis Almpouras (PA)
\item Ioannis Kasidakis (IK)
\item Zafeirios Fountas (ZF)
\item Pedro Martínez Mediano(PMM)
\end{compactenum}

\subsection*{Summary}
This meeting was mainly focused on discussing questions we had after reading all the relevant papers as well as what should be the next steps. A joint meeting took place after which, separate group meetings were conducted with each relevant supervisor to discuss in more detail about the specifics of the upcoming tasks.

\maketitle
\section*{28 January 2015}
\subsection*{Attendance}
\begin{compactenum}
\item Juan Carlos Farah (JCF)
\item Christos Kaplanis (CK)
\item Erik Grabljevec (EG)
\item Panagiotis Almpouras (PA)
\item Ioannis Kasidakis (IK)
\item Zafeirios Fountas (ZF)
\item Pedro Martínez Mediano(PMM)
\end{compactenum}

\subsection*{Summary}
This meeting was mainly focused on the progress of the two groups so far. Most tasks (regarding replicating work that has been done by researchers so that members get comfortable with the concepts and methods) were on track and progressing as expected. A decision was made on the general structure of the first report (due on 06/02/2015) and each member was assigned a part of the report that they would have to prepare before the next meeting.

\maketitle
\section*{04 February 2015}
\subsection*{Attendance}
\begin{compactenum}
\item Juan Carlos Farah (JCF)
\item Christos Kaplanis (CK)
\item Erik Grabljevec (EG)
\item Panagiotis Almpouras (PA)
\item Ioannis Kasidakis (IK)
\item Zafeirios Fountas (ZF)
\item Pedro Martínez Mediano(PMM)
\end{compactenum}

\subsection*{Summary}
This meeting was mainly focused on the first report. With the input of the supervisors and the slides provided by the course instructor Dr. Fidelis Perkonigg the final structure of the report was decided and a soft deadline was set for completion, to allow for a few last amendments before the actual submission on the 06/02/2015.

\maketitle
\section*{11 February 2015}
\subsection*{Attendance}
\begin{compactenum}
\item Juan Carlos Farah (JCF)
\item Christos Kaplanis (CK)
\item Erik Grabljevec (EG)
\item Panagiotis Almpouras (PA)
\item Ioannis Kasidakis (IK)
\item Zafeirios Fountas (ZF)
\item Pedro Martínez Mediano(PMM)
\end{compactenum}

\subsection*{Summary}
This meeting was mainly focused on the progress and issues of each group. The pattern recognition group (JC, EG, PA) had no issues to report and made progress according to plan. The action selection group (CK, IK) upon completion of the initial tasks realised they need an extra component to act as an intermediate between the stimuli input and the action selection neurons. The main issue was the isolation of the small moving targets from much larger objects that may be moving in the background. The action selection group set a goal for next week to identify the most feasible solution by conducting deep background research on the topic.

\maketitle
\section*{18 February 2015}
\subsection*{Attendance}
\begin{compactenum}
\item Juan Carlos Farah (JCF)
\item Christos Kaplanis (CK)
\item Erik Grabljevec (EG)
\item Panagiotis Almpouras (PA)
\item Ioannis Kasidakis (IK)
\item Zafeirios Fountas (ZF)
\item Pedro Martínez Mediano(PMM)
\end{compactenum}

\subsection*{Summary}
This meeting was mainly focused on the progress and issues of each group. The pattern recognition group (JC, EG, PA) had no issues to report and made progress according to plan. The pattern recognition was providing accurate results and was modularised so that it could run independently given a proper input.
\\ The action selection group (CK, IK) upon completion of the initial tasks realised they need an extra component to act as an intermediate between the stimuli input and the action selection neurons. The main issue was the isolation of the small moving targets from much larger objects that may be moving in the background. The action selection group set a goal for next week to identify the most feasible solution by conducting deep background research on the topic.

\maketitle
\section*{25 February 2015}
\subsection*{Attendance}
\begin{compactenum}
\item Juan Carlos Farah (JCF)
\item Christos Kaplanis (CK)
\item Erik Grabljevec (EG)
\item Panagiotis Almpouras (PA)
\item Ioannis Kasidakis (IK)
\item Zafeirios Fountas (ZF)
\item Pedro Martínez Mediano(PMM)
\end{compactenum}

\subsection*{Summary}
This meeting was mainly focused on the action selection mechanism. The action selection group managed to identify a viable solution to the issues mentioned in last week's meeting. An intermediate mechanism should be constructed to approximate the function of the ESTMD neuron. The function of the mechanism would be to pre-process the visual input to isolate small targets. Since the pattern recognition group had almost completed its task one member (EG) was transferred to the action selection group to assist with the development of  ESTMD . A web client was also deemed as a good addition to the project to provide a user friendly interface that connects all the modules together but also allows for independent use of each. The pattern recognition group was further split. One member would focus on the web client (JCF) and the other (PA) would continue working on the pattern recognition mainly creating unit tests to ensure that future changes would not create unspotted issues that affect the robustness and accuracy of the module.

\maketitle
\section*{4 March 2015}
\subsection*{Attendance}
\begin{compactenum}
\item Juan Carlos Farah (JCF)
\item Christos Kaplanis (CK)
\item Erik Grabljevec (EG)
\item Panagiotis Almpouras (PA)
\item Ioannis Kasidakis (IK)
\item Zafeirios Fountas (ZF)
\item Pedro Martínez Mediano(PMM)
\end{compactenum}

\subsection*{Summary}
This meeting was mainly focused on the web client and the ESTMD neuron. A skeleton version of the web client was created by (JCF) and the next step would be to connect the completed pattern recognition module to the web client. Progress was reported on the ESTMD, however the intermediate mechanism revealed some unspotted issues with the CSTMD neuron that is responsible for selecting a single target if multiple stimuli are provided. The action selection group was divided into two sub-groups. Two members (CK, EG) would continue working on the ESTMD and the third member (IK) would focus on the CSTMD to analyse the erroneous behaviour and identify potential solutions. The structure of the second report (due on 13/03/2015) was also discussed.

\maketitle
\section*{11 March 2015}
\subsection*{Attendance}
\begin{compactenum}
\item Juan Carlos Farah (JCF)
\item Christos Kaplanis (CK)
\item Erik Grabljevec (EG)
\item Panagiotis Almpouras (PA)
\item Ioannis Kasidakis (IK)
\item Zafeirios Fountas (ZF)
\item Pedro Martínez Mediano(PMM)
\end{compactenum}

\subsection*{Summary}
This meeting was mainly focused on the second report. With the input of the supervisors and the slides provided by the course instructor Dr. Fidelis Perkonigg the final structure of the report was decided and a soft deadline was set for completion, to allow for a few last amendments before the actual submission on the 13/03/2015.

\maketitle
\section*{18 March 2015}
\subsection*{Attendance}
\begin{compactenum}
\item Juan Carlos Farah (JCF)
\item Christos Kaplanis (CK)
\item Erik Grabljevec (EG)
\item Panagiotis Almpouras (PA)
\item Ioannis Kasidakis (IK)
\item Zafeirios Fountas (ZF)
\item Pedro Martínez Mediano(PMM)
\end{compactenum}

\subsection*{Summary}
This meeting was mainly focused on the web client and the ESTMD neuron. The action selection module was successfully connected to the web client and some prefixed simulations could be run online. Next step would be to add as many functionalities as possible to the web client before the completion and connection of the other modules. Progress was reported on the ESTMD development (CK,EG). No progress was reported on the CSTMD neuron (IK) but a lot of possible ways to address the issue were discussed and would be looked into before the next meeting.

\maketitle
\section*{25 March 2015}
\subsubsection*{No meeting was held this week due to examinations.}

\maketitle
\section*{1 April 2015}
\subsection*{Attendance}
\begin{compactenum}
\item Juan Carlos Farah (JCF)
\item Christos Kaplanis (CK)
\item Erik Grabljevec (EG)
\item Panagiotis Almpouras (PA)
\item Ioannis Kasidakis (IK)
\item Zafeirios Fountas (ZF)
\item Pedro Martínez Mediano(PMM)
\end{compactenum}

\subsection*{Summary}
This meeting was mainly focused on planning the group project activities to take place during the exam preparation period. Members mutually agreed to focus on exams and work on the group project one day a week for the following weeks. Slower but steady progress was expected.  The CSTMD neuron proving much more challenging than originally expected was the only section for which an accurate estimate of the completion date could not be made. For that reason the simpler but similar mechanism of the point neurons was decided to be the last resource in case the CSTMD could not function properly.

\maketitle
\section*{8 April 2015}
\subsubsection*{No meeting was held this week due to Easter holiday.}

\maketitle
\section*{15 April 2015}
\subsection*{Attendance}
\begin{compactenum}
\item Juan Carlos Farah (JCF)
\item Christos Kaplanis (CK)
\item Erik Grabljevec (EG)
\item Panagiotis Almpouras (PA)
\item Ioannis Kasidakis (IK)
\item Zafeirios Fountas (ZF)
\item Pedro Martínez Mediano(PMM)
\end{compactenum}

\subsection*{Summary}
This meeting was mainly focused on discussing the progress of the development of the several modules. The web client was progressing as expected and as soon as each module was completed it would get connected to the web client. The ESTMD was progressing as expected and completed. The CSTMD still could not progress. The target animation along with the re-enforcement learning mechanism were the next things to be tackled. The target animation was assigned to (PA) and (EG), the re-enforcement learning was assigned to (CK). (JCF) would continue working on the web client and (IK) on the CSTMD and the point neurons.

\maketitle
\section*{22 April 2015}
\subsection*{Attendance}
\begin{compactenum}
\item Juan Carlos Farah (JCF)
\item Christos Kaplanis (CK)
\item Erik Grabljevec (EG)
\item Panagiotis Almpouras (PA)
\item Ioannis Kasidakis (IK)
\item Zafeirios Fountas (ZF)
\item Pedro Martínez Mediano(PMM)
\end{compactenum}

\subsection*{Summary}
This meeting was mainly focused on discussing the progress of the development of the several modules. All modules apart from CSTMD were progressing as expected. The CSTMD showed some progress. All members mutually agreed not to work on the group project during the following two weeks as the final exams where taking place during that period.

\maketitle
\section*{29 April 2015}
\subsubsection*{No meeting was held this week due to examinations.}

\maketitle
\section*{6 May 2015}
\subsubsection*{No meeting was held this week due to examinations.}

\maketitle
\section*{8 March 2015}
\subsection*{Attendance}
\begin{compactenum}
\item Juan Carlos Farah (JCF)
\item Christos Kaplanis (CK)
\item Erik Grabljevec (EG)
\item Panagiotis Almpouras (PA)
\item Ioannis Kasidakis (IK)
\item Zafeirios Fountas (ZF)
\item Pedro Martínez Mediano(PMM)
\end{compactenum}

\subsection*{Summary}
This meeting was mainly focused on creating a schedule for the activities to take place during the last week before the submission. (JCF) would work with each member to make the final connection of all the modules to the web client. (IK) would finish the CSTMD neuron asap to test its behaviour. (PA) would start working on the report, creating the main structure and writing any part that can be written without the input of another member (mainly Introduction,  Specification, Group Work, Appendix). A soft deadline was set for completion of the report on the 12/04/2015, to allow for a few last amendments before the actual submission on the 15/04/2015.

\maketitle
\section*{13 May 2015}
\subsection*{Attendance}
\begin{compactenum}
\item Juan Carlos Farah (JCF)
\item Christos Kaplanis (CK)
\item Erik Grabljevec (EG)
\item Panagiotis Almpouras (PA)
\item Ioannis Kasidakis (IK)
\item Zafeirios Fountas (ZF)
\item Pedro Martínez Mediano(PMM)
\end{compactenum}

\subsection*{Summary}
This meeting was mainly focused on the final report. With the input of the supervisors and the materials provided by the course instructor Dr. Fidelis Perkonigg the contents of the report were finalised. A brief discussion on the presentation took place. The presentation would be prepared after the submission of the final report on the 15/04/2015 as the date of the presentation was set to be the 19/05/2015.




\end{document}