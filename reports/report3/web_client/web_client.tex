\documentclass[a4paper,11pt]{article}
\usepackage[margin=2cm]{geometry}

\usepackage[titletoc,toc,title,page]{appendix}
\usepackage[nodayofweek]{datetime}
\usepackage{cite}
\usepackage{graphicx}
\longdate

\usepackage{titlesec}
\usepackage{hyperref}
\usepackage{fancyhdr}
\pagestyle{fancyplain}
\fancyhf{}
\lhead{\fancyplain{}{M.Sc.\ Group Project Report}}
\rhead{\fancyplain{}{\today}}
\cfoot{\fancyplain{}{\thepage}}

\begin{document}

\subsection{Web Client}

The web client is designed to be a simple interface through which simulations for each of the modules can be run and automated, both separately and jointly. We have currently implemented a prototype using Bottle and MongoDB of the interface for the pattern-recognition module. This graphical user interface provides the minimal functionality needed to create sample spike trains, test pattern-recognition neurons against them and save the results of each experiment, providing key insight to the effect of each parameter on the output.

In order to maintain consistency with the rest of the project, we decided to use a web framework for Python that could easily connect with each of our modules. The basic requirements were that the client could be easily deployed on a local environment for testing purposes, but also able to serve multiple users concurrently. We chose Bottle (http://bottlepy.org) as it is a lightweight web framework with a built-in HTTP development server that would address the first requirement out-of-the-box. Additionally it can be seamlessly paired with Nginx (http://wiki.nginx.org/), a high-performance HTTP server, through uWSGI (https://uwsgi-docs.readthedocs.org/en/latest/), a full stack interface between web frameworks and web servers, fulfilling the second requirement.

Given that each of our modules' simulations are computationally expensive, it was imperative that the output and results of each run be saved in a persistent store. As shown in figure X, our project consists of five modules connected sequentially, with the output of the each module in the sequence serving as input to the next module. By saving the output of each module during a simulation, we would be able to perform multiple runs of the next module in the sequence without rerunning the previous modules. Additionally, by saving the data generated by each run, the web client can provide a view of the results for analysis by simple querying the database. We chose MongoDB as a data store.


\end{document}