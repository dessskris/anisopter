\documentclass[a4paper,11pt]{article}
\usepackage[margin=2cm]{geometry}

\usepackage[nodayofweek]{datetime}
\usepackage{cite}
\usepackage{graphicx}
\longdate

\usepackage{hyperref}
\usepackage{fancyhdr}
\pagestyle{fancyplain}
\fancyhf{}
\lhead{\fancyplain{}{M.Sc.\ Group Project Report}}
\rhead{\fancyplain{}{\today}}
\cfoot{\fancyplain{}{\thepage}}


\title{Implementation of attentional bistability of the dragonfly visual neurons in an intelligent biomimetic agent\\\Large{--- Final Report ---}}
\author{Juan Carlos Farah, Panagiotis Almpouras, Ioannis Kasidakis, Erik Grabljevec, Christos Kaplanis\\
       \{jcf214, pa512, ik311, eg1114, ck2714\}@doc.ic.ac.uk\\ \\
       \small{Supervisors: Professor Murray Shanahan, Zafeirios Fountas, Pedro Mediano}\\
       \small{Course: CO530/533, Imperial College London}
}

\begin{document}
\maketitle

\tableofcontents

\clearpage
\section{Introduction}

Dragonflies are insects of the order Odonata, suborder Anisoptera \cite{dfwiki}. They are notoriously effective at prey capture, making the neural processes that underlie this ability particularly interesting to investigate. There has been substantial research into the visual system of dragonflies but what seems to be lacking is an effective tool that links models of the various layers of processing together, which could help us better understand the function of each layer. For example, the centrifugal small target motion detector neuron (CSTMD) is a higher order visual neuron in the brain of the dragonfly. This neuron reacts to the presentation of multiple visual stimuli by firing as if only one of the stimuli was present; this is presumably an attentional selection mechanism \cite{w13}. At Professor Murray Shanahan's lab, researchers have simulated the large contralateral dendritic field of the CSTMD neuron with a biophysical multi-compartmental Hodgkin-Huxley model. Along with Klaus Stiefel \cite{ne13}, they found that with certain numbers of inhibitory synapses and potassium conductance densities, two mutually-coupled CSTMD neurons are capable of a bistable switching process between two input patterns. In order to confirm that this neuron is indeed responsible for target selection, it would be useful to be able to model the CSTMD as part of a whole visual system. Our goal is to create a tool that could hopefully serve the following purposes:
\begin{enumerate}
\item Provide a connected model of dragonfly target selection, starting from the visual input to the retina and ending with the motion of the dragonfly.
\item Provide a user-friendly interface for changing the parameters of the parts of the model and viewing useful information on the processing done by each layer.
\item Provide user-friendly storage of simulations of individual layers or as a whole system.
\item Provide a platform for potentially replicating the behaviour of a real dragonfly during prey capture.
\end{enumerate}


\noindent
\textbf{Report Structure}\\
\noindent
The parts that this report includes are the following:


Specification: The original goals of the project and how they were adjusted according to the challenges that were encountered during the project. Additional goals were also set to ensure a balanced workload of all the members and to achieve the best possible result given the high level of complexity and the time constraints of the project.

Design: An overview of the overall design of the project. The options that were considered and the justification of the choices made for each component individually and for the project as a whole.

Methodology: The methods used to meet the goals set in the specification. The problems and challenges that were faced during the project. The choices made and the software development techniques used to address those issues.

Group work: The division of the project into smaller tasks. The division of the group into sub-groups and the tasks that each sub-group or individual had to complete to lead to the success of this project.

Final product: The overall result of the work conducted throughout the project. The goals that were met as well as the goals that were infeasible given the constraints of the project. Motivation for future development of this project.

Appendix: A log of the summary of the minutes of the meetings that were scheduled from the beginning till the completion of the project. Tasks of each member and overall contribution. 

\clearpage
\section{Specification}

Specification of the functionality that your project aims to provide. The
contents of this section would normally be provided by Report One. In 
addition to that, explain whether you had to make any changes to the original
requirements, and why (this may be taken from Report Two.). 


\clearpage
\section{Design}

This section should provide the overall design of your project.
You should justify your main design choices and discuss other options you have considered. 


\clearpage
\section{Methodology}

This section should describe the method you have used for solving the tasks
described in your specification. How was the overall task divided into different
part-problems? What were the intellectual and/or technical problems that had 
to be solved? What techniques did you consider using for solving the problem? 
Which did you choose and why? What software development techniques have you 
used to conduct your project? Note that some of these issues would have been 
addressed in Report Two.

The following subsections are suggestions \emph{only}.

\subsection{Problems To Bo Solved}

\subsection{Software Engineering Techniques}

\subsection{Division of Tasks}


\clearpage
\section{Group Work}

Division of work; this is not a long story - just a brief statement of who does what.
(Note that in the actual report, the description of the parts and the techniques used
to implement them, and the list of people doing each part might be merged, also note
that your Report Two should already provide this information.) 


\clearpage
\section{Final Product}

Description of the final product you have produced, i.e. your overall achievement.
What have you implemented? Anything not implemented? What were the difficulties?
Any difficulties you managed to overcome? Any results (this would apply more to
investigative projects than to implementation projects)? Testing: how did you test
it and what are the results? Evaluation: How good is your product, how well does
it perform, how accurately does it satisfy the specification? 

The following subsections are suggestions \emph{only}.

\subsection{Final Product Description}

\subsection{Testing \& Results}

\subsection{Product Evaluation
}

\clearpage
\appendix

\section{Logbook}

\clearpage
\section{Minutes of Group Meetings}

\clearpage
\section{Detailed Work Breakdown}


\end{document}

