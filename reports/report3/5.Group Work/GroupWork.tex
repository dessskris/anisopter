\documentclass[a4paper,11pt]{article}
\usepackage[margin=2cm]{geometry}

\usepackage[nodayofweek]{datetime}
\usepackage{cite}
\usepackage{graphicx}
\longdate

\usepackage{hyperref}
\usepackage{fancyhdr}
\pagestyle{fancyplain}
\fancyhf{}
\lhead{\fancyplain{}{M.Sc.\ Group Project Report}}
\rhead{\fancyplain{}{\today}}
\cfoot{\fancyplain{}{\thepage}}


\title{Implementation of attentional bistability of the dragonfly visual neurons in an intelligent biomimetic agent\\\Large{--- Report One ---}}
\author{Juan Carlos Farah, Panagiotis Almpouras, Ioannis Kasidakis, Erik Grabljevec, Christos Kaplanis\\
       \{jcf214, pa512, ik311, eg1114, ck2714\}@doc.ic.ac.uk\\ \\
       \small{Supervisors: Professor Murray Shanahan, Zafeirios Fountas, Pedro Mediano}\\
       \small{Course: CO530/533, Imperial College London}
}

\begin{document}
\maketitle

\section{Group Work}
This section describes the division of the total workload among the group members. Flexibility and adaptation were key in the successful division of the tasks. Juan Carlos Farah assumed the position of the team leader due to his previous working experience in technology related projects.

\subsection{Initial Division}
Initially the team was divided into two subgroups, the action selection and the pattern recognition group. The original estimation was that the action selection task would be much easier than the pattern recognition as for the former, relevant third party code was supplied. Therefore Christos Kaplanis along with Ioannis Kasidakis were assigned to the action selection group and Juan Carlos Farah, Erik Grabljevec and Panagiotis Almpouras were assigned to the pattern recognition group.

\subsection{Reorganising to meet targets}
As mentioned earlier in the report, the CSTMD neuron of the action selection part did not behave as expected. With the addition of the ESTMD neuron and the web client a reform was required to ensure that the project would progress as smoothly as possible. Erik Grablevec was tranferred to the action selection group and along with Christos Kaplanis started working on the ESTMD neuron. Ioannis Kasidakis kept working on the CSTMD neuron trying to solve the issues that had arisen. Juan Carlos Farah started working on the web client. Initially he created the general structure and later in cooperation with each member he connected the individual modules to the web client and with each other. Panagiotis almpouras continued working on the pattern recognition module finalising its dessign and development.\par

Later on, a new reorganisation was required. Christos Kaplanis  and Erik Grablevec after having completed the ESTMD neuron, they were respectively assigned to the re-enforcement learning and the target animation.

\subsection{Reports and Testing}
With the reports and testing a general strategy was decided early on. Each member would be responsible for providing unit tests for the components he worked on. Same for the report each member should write the part for the report which was most relevant to the part he worked on. The more generic parts of each report were assigned the member(s) that were least busy at the time before the submission.






\end{document}

